\footnotesize

\vspace{0.2em} % spacing before the full-width figure

% ---- Full-width TikZ block (outside multicols) ----
\begin{center}
\begin{tikzpicture}
    \node[anchor=west] at (0,0) {
        \begin{minipage}{0.15\textwidth}
        \centering
        \resizebox{\textwidth}{!}{\includegraphics{homer_smart}}
        \end{minipage}
    };
    \node[anchor=west] at (2.5,0) { % Adjust X offset if needed
        \begin{minipage}{0.8\textwidth}
%         \begin{tcolorbox}
% Lorem ipsum dolor sit amet, consectetuer adipiscing elit. Aenean commodo ligula eget dolor. Aenean massa. Cum sociis natoque penatibus et magnis dis parturient montes, nascetur ridiculus mus. Donec quam felis, ultricies nec, pellentesque eu, pretium quis, sem. \textsuperscript{[3]}.
%         \end{tcolorbox}
        The detection of multiple Bog-38 MAGs in the NSPSF challenges the prevailing notion that this archaeal lineage is confined to Arctic and sub-Arctic peatlands. The presence of high-quality, complete MAGs encoding key methanogenesis genes such as the MCR and MTR complexes confirms their potential role in methane production within tropical ecosystems. Phylogenomic comparisons revealed that the tropical Bog-38 representatives form distinct clades, indicating previously unrecognized lineage diversification and suggesting that the biogeographical range and ecological niches of Bog-38 may be broader than currently understood, perhaps remained to be explored in tropical ecosystems.
        
        These findings lay the groundwork for future work to characterize the physiology, environmental responses, and ecological contributions of Bog-38 lineages in methane cycling across contrasting peatland ecosystems. More details available at: \url{github.com/ZarulHanifah/8thICMBB2025}
        \end{minipage}
    };
\end{tikzpicture}
\end{center}