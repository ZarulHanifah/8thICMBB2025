% Put the motivation behind and goals of the research here
\footnotesize
% \lipsum[2]

{\setstretch{0.85}

\textbf{Introduction:} The archaeal class Bog-38 has been predominantly associated with cold-region peatlands, particularly across Arctic and sub-Arctic permafrost ecosystems, where it contributes to methane production. However, its presence and role in tropical environments remain poorly understood. \textbf{Methods:} To investigate the potential distribution and function of Bog-38 in tropical ecosystems, peat soil samples were collected from four distinct sites within the North Selangor Peat Swamp Forest (NSPSF), Peninsular Malaysia. Five different depths were sampled at each site. DNA was extracted from these samples and sequenced using the Oxford Nanopore P2 Solo platform to generate long-read metagenomic data. \textbf{Results and Discussion:} Twelve Bog-38 metagenome-assembled genomes (MAGs) were recovered, with at least three assembled as complete circular genomes. These tropical MAGs contained key methanogenesis genes—mcrA, mcrB, and mcrG—encoding the methyl-coenzyme M reductase (MCR) complex, essential for methane biosynthesis. Comparative phylogenomic analysis revealed that the NSPSF Bog-38 MAGs form distinct clades, indicating significant phylogenetic divergence from Arctic counterparts. Functional profiling of the tropical MAGs also suggested unique metabolic adaptations potentially relevant to the anaerobic, lignin-rich environment of tropical peat soils. \textbf{Conclusion:} These findings expand the known geographic and functional diversity of Bog-38, establishing tropical peatlands like the NSPSF as important reservoirs of novel methanogenic archaea. This challenges previous assumptions of their restriction to permafrost ecosystems and highlights the broader ecological relevance of tropical methane-cycling communities.
}